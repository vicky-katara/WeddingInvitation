\documentclass{standalone}
\usepackage[utf8]{inputenc} 
\usepackage[T1]{fontenc} 
\usepackage[dvipsnames]{xcolor} 
\usepackage[object=vectorian]{pgfornament}
\usepackage{polyglossia}
\usepackage[none]{hyphenat}
\usepackage{caption}
\setdefaultlanguage{english}
\setotherlanguage{sanskrit}
\usetikzlibrary{shapes.geometric,calc}
\definecolor{fondpaille}{cmyk}{0,0,0.1,0}
\newfontfamily\devanagarifont[Scale=MatchUppercase]{Devanagari MT}

\begin{document}
\pagecolor{fondpaille}
\color{Maroon}
\vspace{12pt}
\begin{tikzpicture}[every node/.style={inner sep=0pt}]
\node[text width=12cm,align=center](Text){%
\\
Nisha Chandwani and Vicky Katara joyfully invite your gracious presence at the celebration of their marriage. Please honour us with your company.\\
\vspace{12pt}

\begin{table}[!htb]
\begin{minipage}{.5\linewidth}
\color{Maroon}
\begin{tabular}{ r l }
\multicolumn{2}{c}{\textbf{18th October, 2018}} \\
\hline\\[-1em]
1 PM & Lunch \\
2 PM & Mehendi \\
4 PM & Tea \& Snacks\\
7 PM & Sangeet, Engagement Dinner\\
& \\
\end{tabular}
\end{minipage}%
\color{Maroon}
\hspace{0.2cm}
\begin{minipage}{.5\linewidth}
%\centering
\begin{tabular}{ r l }
\multicolumn{2}{c}{\textbf{19th October, 2018}} \\
\hline\\[-1em]
10.15 AM & Satt Puja (Groom) \\
11.30 AM & Satt Puja (Bride) \\
4.30 PM & Baraat \& Pheras\\
5 PM & Tea \& Snacks\\
8 PM & Reception Dinner\\
\end{tabular}
\end{minipage}
\end{table}

\color{Maroon}

\vspace{12pt}
\textbf{Location}: 
Avion Hotel\\
Mumbai Domestic Airport\\
Airport Road, Vile Parle East\\
Mumbai\\
\vspace{12pt}
\begin{tikzpicture}
\node (A) at (-4,0) {};  
\draw [fill=Maroon!20]  (A) circle (2pt);
\end{tikzpicture}

\vspace{12pt}
\foreignlanguage{sanskrit}{
निशा चंदवानी और विक्की कटारा खुशी-खुशी उनकी शादी के शुभ उत्सव पर आपको आमंत्रित करते हैं। वे उम्मीद करते हैं कि आप उन्हें अपनी उपस्थिति से आशीर्वाद देंगे। \\}
\vspace{12pt}

\begin{table}[!htb]
	\begin{minipage}{.5\linewidth}
		\begin{tabular}{ r l }
			\multicolumn{2}{c}{\foreignlanguage{sanskrit}{अक्टूबर १८, २०१८}} \\
			\hline\\[-1em]
			\foreignlanguage{sanskrit}{दोपहर १ बजे} & \foreignlanguage{sanskrit}{भोजन} \\
			\foreignlanguage{sanskrit}{दोपहर २ बजे} & \foreignlanguage{sanskrit}{मेहंदी} \\
			\foreignlanguage{sanskrit}{शाम ४ बजे} & \foreignlanguage{sanskrit}{चाय और नाश्ता}\\
			\foreignlanguage{sanskrit}{शाम ७ बजे} & \foreignlanguage{sanskrit}{संगीत, सगाई और खाना}\\
			& \\
		\end{tabular}
	\end{minipage}%
	\begin{minipage}{.5\linewidth}
		\begin{tabular}{ r l }
			\multicolumn{2}{c}{\foreignlanguage{sanskrit}{अक्टूबर १९, २०१८}} \\
			\hline\\[-1em]
			\foreignlanguage{sanskrit}{सुबह १०.१५ बजे} & \foreignlanguage{sanskrit}{सत्त पूजा (दूल्हा)} \\
			\foreignlanguage{sanskrit}{सुबह ११.३० बजे} & \foreignlanguage{sanskrit}{सत्त पूजा (दुल्हन)} \\
			\foreignlanguage{sanskrit}{शाम ४.३० बजे} & \foreignlanguage{sanskrit}{बारात और फेरे} \\
			\foreignlanguage{sanskrit}{शाम ५ बजे} & \foreignlanguage{sanskrit}{चाय और नाश्ता} \\
			\foreignlanguage{sanskrit}{शाम ८ बजे} & \foreignlanguage{sanskrit}{रिसेप्शन} \\
		\end{tabular}
	\end{minipage}
\end{table}




\vspace{12pt}
\foreignlanguage{sanskrit}{\textbf{स्थान}}: \foreignlanguage{sanskrit}{एवियन होटेल}\\
\foreignlanguage{sanskrit}{मुंबई डोमेस्टिक एयरपोर्ट}\\
\foreignlanguage{sanskrit}{वीले पार्ले ईस्ट}\\
\foreignlanguage{sanskrit}{मुंबई}\\
} ;
\node[shift={(-1cm,1cm)},anchor=north west](CNW)  at (Text.north west)
               {\pgfornament[width=2cm]{61}};
\node[shift={(1cm,1cm)},anchor=north east](CNE)   at (Text.north east)
               {\pgfornament[width=2cm,symmetry=v]{61}}; 
\node[shift={(-1cm,-1cm)},anchor=south west](CSW) at (Text.south west)
               {\pgfornament[width=2cm,symmetry=h]{61}}; 
\node[shift={(1cm,-1cm)},anchor=south east](CSE)  at (Text.south east)   
               {\pgfornament[width=2cm,symmetry=c]{61}};  
\pgfornamenthline{CNW}{CNE}{north}{87}
\pgfornamenthline{CSW}{CSE}{south}{87}
\pgfornamentvline{CNW}{CSW}{west}{87}
\pgfornamentvline{CNE}{CSE}{east}{87}
\end{tikzpicture}
\vspace{12pt}
\end{document}